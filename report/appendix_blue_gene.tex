\chapter{Overview of Blue Gene/L System}
The basic building block of the Blue Gene/L is called the computer chip. Each computer chip consists of two PPC 440 cores, with a 32 KB L1 cache and a 2 KB L2 cache. The cores share a 4 MB EDRAM L3 cache. A compute card contains two computer chips. A node card contains 16 compute cards, and a midplane holds 16 node cards with a total of 1024 processors. A midplane also contains several I/O nodes which are configured to handle file I/O and host communication. Each midplane also has one service card that performs system management services. These system management services include activitis such as monitoring node heartbeat and checking errors. 

In Blue Gene, the Cluster Monitoring and Control System (CMCS) service is implemented on the service nodes. The purpose of the CMCS is system monitoring and error checking. Specific device information is acquired by the service node, which is available on each midplane. This information includes RAS(reliability, availability and serviceability) events and other useful data, and is acquired by service nodes directly through the control network. A polling agent is used to collect runtime information from the computer and I/O nodes. This information is reported to the CMCS service, after which it is stored in a centralized DB2 repository. The granularity of this system event logging mechanism is less less than 1 ms.
\section{RAS Logs}
RAS stands for Rliability, Availability and Servicability. RAS logs are machine logs recorded by IBM machines and contain error traces and informational records . RAS logs have been commonly used for machine failure analysis. RAS messages for daemons that are down and jobs that are being rejected or vacated are always logged. The project uses RAS logs of the IBM Blue Gene/L machine recorded over six months to train the fault prediction mechanism and also to revise it. A sample RAS. A sample record from the Blue Gene/L RAS log is as follows:
\begin{table}[h]
\begin{center}
\small
\begin{tabular}{| p{1cm} | p{2.5cm} | p{1.5cm} | p{2cm} | c | c | p{3cm} |}
\hline	
	\bf{recid} & \bf{event\_time} & \bf{location} & \bf{event\_type} & \bf{facility} & \bf{severity} & \bf{entry\_data} \\ \hline
	1 & 2005-06-03-15.42.50.363779 & R02-M1-N0-C:J12-U11 & RAS & KERNEL & INFO & instruction cache parity error corrected \\ \hline
\end{tabular}
\label{table2}
\end{center}
	\caption{Sample Blue Gene/L RAS Log Entry.}
\end{table}
\begin{itemize}
	\item recid - Recid or record ID is used to uniquely identity a record in the RAS log.
	\item Location - Specifies the identifier used to identify the node card that reported the message.
	\item event\_type - Specifies the type of the event mostly RAS.
	\item Facility - Specifies the source from which the message originated.
	\item Severity - Specifies the severity of the message whether fatal, failure, non fatal informational message or severe.
	\item entry\_data - Specifies the actual message that was generated by the Blue Gene/L machine when the corresponding event occurred.
\end{itemize}
